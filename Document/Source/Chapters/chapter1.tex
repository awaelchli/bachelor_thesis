\chapter{Introduction}
\label{chp:introduction}

\todo{Possible figures to insert: Lytro camera, Stanford camera array, Tensor Display etc.}

Over the last few years, devices capable of displaying 3D content have shown to become increasingly popular.
Most of the moviegoers have long accustomed to the variety of movies releasing in 3D every year, and with affordable 3D television screens on the market, movies with the extra dimension can be enjoyed in the living room.
Current graphics processors are powerful enough to bring the 3D experience to the video game consumer, allowing for a higher immersion into the virtual world.

There are two main categories of such displays, stereoscopic- and true 3D displays.

\section{Stereoscopic Displays}

Stereoscopic displays are based on the principles of binocular vision.
The objective is to provide two distinct images to the human visual system, one for each eye, presenting the content from two slightly different perspectives.
The disparities between the two images translate to depth cues in the human brain and allow for depth perception.
The pair of images presented to the eyes remains constant when the viewer moves in front of the device. 
This effect distinguishes stereoscopic displays from 3D displays.
Modern technologies include head-mounted displays, polarization systems, active shutter systems and autostereoscopy.
Although not as comfortable to wear, head-mounted displays have separate high resolution screens for each eye allowing for a high degree of immersion.
Polarization screens show the image pair superimposed with different polarization of the light, which is separated again by different polarization filters in the right and left side of the viewers eyeglasses.
Active shutter systems use special eyeglasses that alternately block the light for one eye, letting the opposite eye see the corresponding image on the synchronized screen.
Autostereoscopic displays present stereo content to the viewer without the need of special glasses. 
The technology is based on a lenticular lens or parallax barriers, which requires the viewer to be in a fixed and predefined position. 

\section{3D Displays}

Real 3D displays ideally show the full 3D information to the observer.
In contrast to stereoscopic displays, the person is able to move in front of the screen and view the content from a desired perspective.
Present technologies include volumetric displays, holography, integral imaging and compressive light field displays.
Volumetric displays reproduce a physical volume emitting the light of virtual objects inside, allowing for a full 360 degree viewing angle.%without the need of special glasses.
Holographic displays are based on conventional LCD panels equipped with a diffraction layer making it possible to project images in different directions in space.
Integral imaging devices achieve the same result with a microlens array in front of the screen similar to lenticular lenses.
Finally, compressive light field displays, also called tensor displays, consist of multiple LCD panels forming a stack of time multiplexed, light attenuating layers.

The work in this thesis is based on a much simpler version of these light field displays, called \emph{Layered 3D}, which was first realized by~\cite{WetzsteinTomo}.
The display is able to present a static, full 4D light field without the need of special glasses.
It consists of masks, printed on transparencies, which attenuate light from a backlight in a multiplicative manner, the same concept tensor displays are based on.

\begin{figure}
	\centering
	\subfigure[]{
		\includegraphics[scale = 0.4]{placeholder}
		\label{fig:attenuation_layers_and_glasses}
	}
	%\hfill
	\subfigure[]{
		\includegraphics[scale = 0.4]{placeholder}
		\label{fig:close_up_of_layers_between_glasses}
	}
	\caption[Attenuation layers between glass plates]
			{Attenuation layers between glass plates.
			 (a) Front view of the display.
			 (b) Side view: Ten pieces of 2 mm thick glass plates hold the five layers of transparencies, with a 4 mm separation between them.}
\end{figure}

\section{Light Fields}




\section{Related Work}
