\thispagestyle{empty}
\vspace{8cm}
\noindent
{\centerline {\bf \large Abstract}}
\vspace{1cm}
\noindent

This work reviews and extends an existing, glasses-free 3D display called \emph{Layered 3D} which is based on light field technology.
The purpose of the thesis is to understand this specific multi-layer architecture in terms of spectral properties and computational limitations and to adapt the previous work to support non-synthetic light fields.
In the first part, the theoretical concept of light attenuation inside the display is explained and presented as an optimization problem.
The problem of finding the correct attenuation values is closely related to computed tomography and as such, a tomographic reconstruction technique is applied.
Subsequently, the limitations of depth of field are presented with an analysis on the spectral support of the light field display.
The second part covers the practical implementation of the software as well as the physical realization of an attenuation display with backlight.
The developed software provides features that assist the user with importing a light field, finding optimal display parameters, simulating the projected images and finally, printing the attenuation masks on transparencies.