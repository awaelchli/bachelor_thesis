\documentclass[tikz]{standalone}

\usetikzlibrary{intersections}

\begin{document}
	\begin{tikzpicture}[scale = 0.35, baseline]
		
		% First layer axis
		\draw[name path = layer1, ->] (-1, 3) --(11, 3) node[right] {Layer 2};
		
		% Second layer axis
		\draw[name path = layer2, ->] (-1, -1) --(11, -1) node[right] {Layer 1};
		
		% Pixel boundaries on first layer
		\foreach \i in {1,...,4} 
			\draw ({(\i - 1) * 3}, 3) -- ++(0, 0.5) -- ++(0, -1);

		% Pixel boundaries on second layer
		\foreach \i in {1,...,4} 
			\draw ({(\i - 1) * 3}, -1) -- ++(0, 0.5) -- ++(0, -1);	
			
		% Pixel centers on first layer	
		\foreach \i in {1,...,3} 
			\fill ({1.5 + (\i - 1) * 3}, 3) circle[radius = 0.2];	
		
		% Pixel centers on second layer	
		\foreach \i in {1,...,3} 
			\fill ({1.5 + (\i - 1) * 3}, -1) circle[radius = 0.2];	
			
		% The ray passing the two layers
		\begin{scope}
			\clip (0, 5) rectangle (11, -3);
			\coordinate (anchor) at (2.3, -1);
			
			\draw[name path = ray, red] (anchor) -- ++(46 : 10cm);
			\draw[red, ->] (anchor) -- ++(180 + 46 : 2cm);
		\end{scope}
		
		% Mark intersections with layers
		\fill[red] (anchor) circle[radius = 0.2];
		\fill[name intersections={of=layer1 and ray, total=\t}, red]
			\foreach \s in {1,...,\t}{(intersection-\s) circle[radius = 0.2]};
			
		% Interpolation markers and labels on first layer
		\draw[name intersections={of=layer1 and ray, total=\t}, blue, |-|]
			(intersection-1) ++(0, -1) -- node[below, yshift = -1.5] {$1 - \gamma_2$} (7.5, 2);
		
		\draw[name intersections={of=layer1 and ray, total=\t}, blue, |-|]
			(intersection-1) ++(0, 1) -- node[above, yshift = 1.5] {$\gamma_2$} (4.5, 4);
			
		% Interpolation markers and labels on second layer
		\draw[blue, |-|]
			(anchor) ++(0, 1) -- node[above, yshift = 1.5] {$\gamma_1$} (1.5, 0);
				
		\draw[blue, |-|]
			(anchor) ++(0, -1) -- node[below, yshift = -1.5] {$1 - \gamma_1$} (4.5, -2);
		
	\end{tikzpicture}
\end{document}
