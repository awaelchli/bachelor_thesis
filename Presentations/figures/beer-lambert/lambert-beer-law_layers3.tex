% Lambert-Beer law parameters drawing
% Author: Michele Muccioli
% Compile with LuaLaTeX
\documentclass{standalone}
\usepackage{tikz}
\usetikzlibrary{calc,fadings,decorations.markings}
\usepackage{amsmath}
%%%<
\usepackage{verbatim}
%%%>
\begin{comment}
:Title: Lambert-Beer law parameters drawing
:Tags: Foreach; Physics; Optics
:Author: Michele Muccioli
:Slug: lambert-beer-law
The Lambert-Beer law states there is a logarithmic dependence
between the ratio between incident light intensity $I_0$ and
the intensity of scattered light directing forward $I$ through
a media and the product of the absorption coefficient $\gamma$
of the substance and the distance the light travels through
the material $L$.
\[
\dfrac{I}{I_0}=\exp(-\gamma L)
\]
In the case of the figure above, the light is a red laser beam
through a cloud formed by sand grains with a certain granulometry
schematized with spherical particles.

To deep the argument ($\gamma$ computation), see ``Light scattering
by small particles'' by Hulst van de H.C., Dover Publications.
\end{comment}

%%%%%%%%%%%%%%%%%%%%%%%%%%%
% FADING LIGHT DECORATION %
%%%%%%%%%%%%%%%%%%%%%%%%%%%
\makeatletter
\pgfkeys{/pgf/decoration/.cd,
         start color/.store in = \startcolor,
         end color/.store in   = \endcolor
        }

\pgfdeclaredecoration{color change}{initial}{
% Initial state
\state{initial}[%
    width                     = 0pt,
    next state                = line,
    persistent precomputation = {\pgfmathdivide{50}{\pgfdecoratedpathlength}%
                                 \let\increment=\pgfmathresult%
                                 \def\x{0}}]%
{}%

% Line state
\state{line}[%
    width                      = .5pt,
    persistent postcomputation = {\pgfmathadd@{\x}{\increment}%
                                  \let\x=\pgfmathresult}]%
{%
  \pgfsetlinewidth{\pgflinewidth}%
  \pgfsetarrows{-}%
  \pgfpathmoveto{\pgfpointorigin}%
  \pgfpathlineto{\pgfqpoint{.75pt}{0pt}}%
  \pgfsetstrokecolor{\endcolor!\x!\startcolor}%
  \pgfusepath{stroke}%
}%

% Final state
\state{final}{%
  \pgfsetlinewidth{\pgflinewidth}%
  \pgfpathmoveto{\pgfpointorigin}%
  \color{\endcolor!\x!\startcolor}%
  \pgfusepath{stroke}% 
}
}
\makeatother

%%%%%%%%%%%%
% COMMANDS %
%%%%%%%%%%%%
\def\pr#1{\directlua{tex.print(#1)}}

\def\rnd{.%
\pdfuniformdeviate10%
\pdfuniformdeviate10%
\pdfuniformdeviate10%
}

\begin{document}
%%%%%%%%%%%%%%
% PARAMETERS %
%%%%%%%%%%%%%%
\definecolor{sand}{RGB}{193,154,107} % Particles color
\def\cols{20}                        % Number of columns
\def\rows{40}                        % Number of rows
\def\SquareUnit{.35}                 % Lengths of unit square edges (cm)
\pgfmathsetmacro\RmaxParticle{.1}    % Maximum particle radius
\def\BeforeLight{5}                  % Light path before particle cloud
\begin{tikzpicture}[x          = \SquareUnit cm,
                    y          = \SquareUnit cm,
                    line width = 2pt
                   ]
                               
%%%%%%%%%%%%%%
% LIGHT PATH %
%%%%%%%%%%%%%%
%-> Before particles cloud
\draw[red,
      decoration = {markings,
                    mark = at position 0.5 with {\arrow[]{latex}}},
      postaction = {decorate}] (-\BeforeLight,{\rows*\SquareUnit/2})--++
                               (\BeforeLight,0)node[midway,
                                                    above,
                                                    black]{$I_0$};

%-> Trespassing particles cloud
\draw[decoration = {color change,
                    start color = red,
                    end color   = red!20!white},
                    decorate] (0,{\rows*\SquareUnit/2})--++
                              (\cols*\SquareUnit,0);

%-> After particles cloud
\draw[red!20!white,
      decoration = {markings,
                    mark = at position 0.5 with {\arrow[]{latex}}},
      postaction = {decorate}] ({\cols*\SquareUnit},{\rows*\SquareUnit/2})--++
                               (\BeforeLight,0)node[midway,
                                                    above,
                                                    black]{$I$};
%%%%%%%%%%%%%%%%%%%
% PARTICLES CLOUD %
%%%%%%%%%%%%%%%%%%%
%-> Lua version (FASTER)
% LAYERS %

\fill[opacity = 0.3] (0, 0) rectangle ++(0.4, 14);
\fill[opacity = 0.3] (1.5, 0) rectangle ++(0.4, 14);
\fill[opacity = 0.3] (3, 0) rectangle ++(0.4, 14);
\fill[opacity = 0.3] (6, 0) rectangle ++(0.4, 14);

\fill[draw = black, fill=white, line width = 0.3mm] (-5, 0) rectangle ++(-0.4, 14);


\foreach \i in {1,...,1}{
	\foreach \j in {1,...,\rows}{
        \edef\radius{\pr{\RmaxParticle*math.random()}}
        \edef\l{\pr{\SquareUnit-2*\radius}}
        \edef\x{\pr{(\i-1)*\SquareUnit+\radius+\l*math.random()}}
        \edef\y{\pr{(\j-1)*\SquareUnit+\radius+\l*math.random()}}
        \fill[sand] (\x,\y)circle[radius=\radius];
        }
}

\foreach \i in {1,...,1}{
	\foreach \j in {1,...,\rows}{
		\edef\radius{\pr{\RmaxParticle*math.random()}}
		\edef\l{\pr{\SquareUnit-2*\radius}}
		\edef\x{\pr{(\i-1)*\SquareUnit+\radius+\l*math.random()}}
		\edef\y{\pr{(\j-1)*\SquareUnit+\radius+\l*math.random()}}
		\fill[sand] (\x + 1.5,\y)circle[radius=\radius];
	}
}

\foreach \i in {1,...,1}{
	\foreach \j in {1,...,\rows}{
		\edef\radius{\pr{\RmaxParticle*math.random()}}
		\edef\l{\pr{\SquareUnit-2*\radius}}
		\edef\x{\pr{(\i-1)*\SquareUnit+\radius+\l*math.random()}}
		\edef\y{\pr{(\j-1)*\SquareUnit+\radius+\l*math.random()}}
		\fill[sand] (\x + 3,\y)circle[radius=\radius];
	}
}

\foreach \i in {1,...,1}{
	\foreach \j in {1,...,\rows}{
		\edef\radius{\pr{\RmaxParticle*math.random()}}
		\edef\l{\pr{\SquareUnit-2*\radius}}
		\edef\x{\pr{(\i-1)*\SquareUnit+\radius+\l*math.random()}}
		\edef\y{\pr{(\j-1)*\SquareUnit+\radius+\l*math.random()}}
		\fill[sand] (\x + 6,\y)circle[radius=\radius];
	}
}


%-> pgfmath version (uncomment it if you want to try)
% Some time compilation gives too high
% number computation problem
%\foreach \i in {1,...,\cols}{
%	\foreach \j in {1,...,\rows}{
%        \pgfmathsetmacro\radius{\RmaxParticle*\rnd}
%        \pgfmathsetmacro\l{\SquareUnit-2*\radius}
%        \pgfmathsetmacro\x{(\i-1)*\SquareUnit+\radius+\l*\rnd}
%        \pgfmathsetmacro\y{(\j-1)*\SquareUnit+\radius+\l*\rnd}
%        \fill[sand] (\x,\y)circle[radius=\radius];
%        }
%}

%%%%%%%%%%%%%%%%%%
% LENGTH QUOTING %
%%%%%%%%%%%%%%%%%%
%\draw[|<->|,
%      >          = latex,
%      line width = .8pt] ($(0,\rows*\SquareUnit)+(0,1)$)--++
%                         (\cols*\SquareUnit,0)node[midway,
%                                                   above]{$l$};
%%%%%%%%%%%%%%%%%%%%
% LAMBERT-BEER LAW %
%%%%%%%%%%%%%%%%%%%%
\node[anchor    = north,
      inner sep = 1ex] at (current bounding box.south){$\dfrac{I}{I_0}=\text{exp} \displaystyle \left(-\int_{\mathcal{R}} \mu (r) \text{d}r \right) \cong \displaystyle \prod_{i} t^{(i)} = \text{exp} \left( \displaystyle - \sum_{i} a^{(i)} \right)$};
      
      
\node[anchor = center] at (0 + 0.3, 6) {$t^{(1)}$};
\node[anchor = center] at (1.5 + 0.3, 6) {$t^{(2)}$};
\node[anchor = center] at (3 + 0.3, 6) {$t^{(3)}$};
\node[anchor = center] at (6 + 0.3, 6) {$t^{(n)}$};

\node[anchor = south] at (4.7, 7) {$\cdots$};

\end{tikzpicture}
\end{document}
